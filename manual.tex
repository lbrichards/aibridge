\documentclass{article}
\usepackage[utf8]{inputenc}
\usepackage{minted}
\usepackage{hyperref}
\usepackage{geometry}
\usepackage{enumitem}

\title{AI Bridge User Manual}
\author{OpenHands}
\date{\today}

\begin{document}
\maketitle

\section{Overview}
AI Bridge is a system that enables AI assistants to execute commands on a local machine and see their output in real-time. It consists of three main components:
\begin{itemize}
    \item AI Bridge - A web service that receives commands
    \item Redis - A message broker for command distribution
    \item Relay - A CLI tool that executes commands and displays output
\end{itemize}

\section{Installation}

\subsection{Prerequisites}
\begin{itemize}
    \item Docker and Docker Compose
    \item Python 3.7+
    \item Redis server
    \item pipx (recommended for tool installation)
\end{itemize}

\subsection{Installing Components}

\subsubsection{AI Bridge}
Clone and build the Docker container:
\begin{minted}{bash}
git clone https://github.com/username/aibridge.git
cd aibridge
docker-compose up --build
\end{minted}

\subsubsection{Relay Tool}
Install using pipx:
\begin{minted}{bash}
cd relay
pipx install -e .
\end{minted}

\section{Usage}

\subsection{Starting the System}

\subsubsection{Start Redis}
On macOS:
\begin{minted}{bash}
brew services start redis
\end{minted}

\subsubsection{Start AI Bridge}
From the aibridge directory:
\begin{minted}{bash}
docker-compose up
\end{minted}

\subsubsection{Start Relay}
From any terminal:
\begin{minted}{bash}
relay start
\end{minted}

\section{Architecture}

\subsection{Command Flow}
\begin{enumerate}
    \item AI Assistant sends command to AI Bridge endpoint
    \item AI Bridge publishes command to Redis
    \item Relay receives command from Redis
    \item Relay executes command and displays output
    \item AI Assistant sees output through tmate session
\end{enumerate}

\section{Configuration}

\subsection{AI Bridge}
Configuration in docker-compose.yml:
\begin{minted}{yaml}
version: '3.8'
services:
  command-share:
    build: .
    ports:
      - "51753:51753"
    environment:
      - REDIS_HOST=host.docker.internal
    extra_hosts:
      - "host.docker.internal:host-gateway"
\end{minted}

\subsection{Relay}
Configuration in pyproject.toml:
\begin{minted}{toml}
[project]
name = "relay-cli"
version = "0.1.0"
dependencies = [
    "redis",
    "typer",
    "colorama"
]
\end{minted}

\section{Troubleshooting}

\subsection{Common Issues}
\begin{itemize}
    \item \textbf{Relay not responding}: Restart relay with \mintinline{bash}{relay start}
    \item \textbf{Redis connection error}: Ensure Redis is running with \mintinline{bash}{brew services start redis}
    \item \textbf{Complex commands failing}: Break complex commands into simpler steps
\end{itemize}

\subsection{Debugging}
\begin{itemize}
    \item Check Redis connection: \mintinline{bash}{redis-cli ping}
    \item Verify AI Bridge is running: \mintinline{bash}{docker ps}
    \item Check relay status: Look for output in relay terminal
\end{itemize}

\section{Best Practices}

\subsection{Command Execution}
\begin{itemize}
    \item Use simple, atomic commands when possible
    \item Avoid complex pipes and redirections
    \item Check command output before proceeding
\end{itemize}

\subsection{System Management}
\begin{itemize}
    \item Restart components in order: Redis → AI Bridge → Relay
    \item Monitor token usage for cost efficiency
    \item Keep commands focused and specific
\end{itemize}

\section{Limitations}
\begin{itemize}
    \item Complex shell operations may require breaking into steps
    \item Token usage increases with command complexity
    \item System requires manual restart if components get out of sync
\end{itemize}

\section{Future Improvements}
\begin{itemize}
    \item Better handling of complex shell commands
    \item Automatic recovery from errors
    \item More efficient token usage
    \item Improved command batching
\end{itemize}

\end{document}