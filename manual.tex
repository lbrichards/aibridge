\documentclass{article}
\usepackage[utf8]{inputenc}
\usepackage{minted}
\usepackage{hyperref}
\usepackage{geometry}
\usepackage{enumitem}

\title{AI Bridge User Manual}
\author{Larry Ando}
\date{\today}

\begin{document}
\maketitle

\section{Overview}
AI Bridge is a system designed to solve a fundamental challenge in AI-human interaction: enabling AI assistants to directly interact with a user's local machine while maintaining security and control. The system addresses several key problems in AI-human collaboration:

\subsection{Problems Solved}
\begin{itemize}
    \item \textbf{Copy-Paste Overhead}: Traditional interaction requires users to manually copy-paste commands and outputs between their terminal and the AI interface, creating friction and potential for errors.
    
    \item \textbf{Command Output Visibility}: AI assistants typically can't see the output of commands executed on the user's machine, making it difficult to provide context-aware assistance or troubleshoot issues.
    
    \item \textbf{Security Concerns}: Direct SSH access for AI assistants raises security concerns, necessitating a controlled interface for command execution.
    
    \item \textbf{Context Maintenance}: The need to constantly switch between contexts (AI chat, terminal, documentation) reduces productivity and increases the chance of mistakes.
\end{itemize}

\subsection{Solution Architecture}
The system consists of three main components working together to provide a seamless interface:

\begin{itemize}
    \item \textbf{AI Bridge} - A web service that acts as a secure gateway for command execution requests. It provides a controlled HTTP API that the AI assistant can use to send commands, respecting security boundaries while enabling automation.
    
    \item \textbf{Redis} - A message broker that handles reliable command distribution. It decouples the command submission from execution, providing a robust and scalable communication channel between components.
    
    \item \textbf{Relay} - A CLI tool that executes commands and displays output. It provides real-time feedback to both the user and the AI assistant, creating a shared context for collaboration.
\end{itemize}

\subsection{Benefits}
This architecture provides several key advantages:

\begin{itemize}
    \item \textbf{Reduced Friction}: Commands can be executed with minimal user intervention, streamlining the interaction between AI and user.
    
    \item \textbf{Shared Context}: Both AI and user can see command outputs in real-time, enabling more effective collaboration and troubleshooting.
    
    \item \textbf{Controlled Access}: The system provides a secure way for AI to interact with the local machine without requiring direct system access.
    
    \item \textbf{Audit Trail}: All commands and their outputs are visible and can be logged, providing transparency and accountability.
\end{itemize}

\section{Installation}

\subsection{Prerequisites}
\begin{itemize}
    \item Docker and Docker Compose
    \item Python 3.7+
    \item Redis server
    \item pipx (recommended for tool installation)
\end{itemize}

\subsection{Installing Components}

\subsubsection{AI Bridge}
Clone and build the Docker container:
\begin{minted}{bash}
git clone https://github.com/username/aibridge.git
cd aibridge
docker-compose up --build
\end{minted}

\subsubsection{Relay Tool}
Install using pipx:
\begin{minted}{bash}
cd relay
pipx install -e .
\end{minted}

\section{Usage}

\subsection{Starting the System}

\subsubsection{Start Redis}
On macOS:
\begin{minted}{bash}
brew services start redis
\end{minted}

\subsubsection{Start AI Bridge}
From the aibridge directory:
\begin{minted}{bash}
docker-compose up
\end{minted}

\subsubsection{Start Relay}
From any terminal:
\begin{minted}{bash}
relay start
\end{minted}

\section{Architecture}

\subsection{System Design Philosophy}
The architecture of AI Bridge is built around several key principles:

\begin{itemize}
    \item \textbf{Separation of Concerns}: Each component has a specific, well-defined role:
    \begin{itemize}
        \item AI Bridge handles command reception and validation
        \item Redis manages message distribution and queuing
        \item Relay focuses on command execution and output display
    \end{itemize}
    
    \item \textbf{Security by Design}: The system uses:
    \begin{itemize}
        \item HTTP endpoints for controlled access
        \item Redis for message isolation
        \item Local execution for command control
    \end{itemize}
    
    \item \textbf{Real-time Feedback}: The system provides:
    \begin{itemize}
        \item Immediate command echo
        \item Live output streaming
        \item Status updates and error messages
    \end{itemize}
\end{itemize}

\subsection{Information Flow}
The system manages several types of information:

\begin{itemize}
    \item \textbf{Commands}: Sent from AI to user's machine
    \begin{itemize}
        \item Transmitted as HTTP requests to AI Bridge
        \item Queued in Redis for reliability
        \item Executed by Relay in user's environment
    \end{itemize}
    
    \item \textbf{Output}: Command results and system messages
    \begin{itemize}
        \item Captured by Relay in real-time
        \item Displayed in user's terminal
        \item Visible to AI through tmate session
    \end{itemize}
    
    \item \textbf{State}: System status and session information
    \begin{itemize}
        \item Managed by Redis for component coordination
        \item Monitored by Relay for system health
        \item Accessible to AI for context awareness
    \end{itemize}
\end{itemize}

\subsection{Command Flow}
The typical flow of a command through the system:

\begin{enumerate}
    \item \textbf{Initiation}: AI Assistant sends command to AI Bridge endpoint
    \begin{itemize}
        \item Uses HTTP POST request
        \item Includes command text and parameters
        \item Receives immediate acknowledgment
    \end{itemize}
    
    \item \textbf{Distribution}: AI Bridge publishes command to Redis
    \begin{itemize}
        \item Validates command format
        \item Adds to message queue
        \item Ensures reliable delivery
    \end{itemize}
    
    \item \textbf{Execution}: Relay receives and processes command
    \begin{itemize}
        \item Subscribes to Redis channel
        \item Executes in local shell
        \item Captures all output streams
    \end{itemize}
    
    \item \textbf{Feedback}: Output displayed and captured
    \begin{itemize}
        \item Shows in user's terminal
        \item Visible in tmate session
        \item Available for AI analysis
    \end{itemize}
\end{enumerate}

\subsection{Error Handling}
The system includes comprehensive error management:

\begin{itemize}
    \item \textbf{Connection Issues}
    \begin{itemize}
        \item Redis connection monitoring
        \item Automatic reconnection attempts
        \item Clear error messages
    \end{itemize}
    
    \item \textbf{Command Failures}
    \begin{itemize}
        \item Execution error capture
        \item Status code reporting
        \item Detailed error messages
    \end{itemize}
    
    \item \textbf{System Recovery}
    \begin{itemize}
        \item Component restart capabilities
        \item State recovery procedures
        \item User notification system
    \end{itemize}
\end{itemize}

\section{Configuration}

\subsection{AI Bridge}
Configuration in docker-compose.yml:
\begin{minted}{yaml}
version: '3.8'
services:
  command-share:
    build: .
    ports:
      - "51753:51753"
    environment:
      - REDIS_HOST=host.docker.internal
    extra_hosts:
      - "host.docker.internal:host-gateway"
\end{minted}

\subsection{Relay}
Configuration in pyproject.toml:
\begin{minted}{toml}
[project]
name = "relay-cli"
version = "0.1.0"
dependencies = [
    "redis",
    "typer",
    "colorama"
]
\end{minted}

\section{Troubleshooting}

\subsection{Common Issues}
\begin{itemize}
    \item \textbf{Relay not responding}: Restart relay with \mintinline{bash}{relay start}
    \item \textbf{Redis connection error}: Ensure Redis is running with \mintinline{bash}{brew services start redis}
    \item \textbf{Complex commands failing}: Break complex commands into simpler steps
\end{itemize}

\subsection{Debugging}
\begin{itemize}
    \item Check Redis connection: \mintinline{bash}{redis-cli ping}
    \item Verify AI Bridge is running: \mintinline{bash}{docker ps}
    \item Check relay status: Look for output in relay terminal
\end{itemize}

\section{Best Practices}

\subsection{Command Execution}
\begin{itemize}
    \item Use simple, atomic commands when possible
    \item Avoid complex pipes and redirections
    \item Check command output before proceeding
\end{itemize}

\subsection{System Management}
\begin{itemize}
    \item Restart components in order: Redis → AI Bridge → Relay
    \item Monitor token usage for cost efficiency
    \item Keep commands focused and specific
\end{itemize}

\section{Limitations}
\begin{itemize}
    \item Complex shell operations may require breaking into steps
    \item Token usage increases with command complexity
    \item System requires manual restart if components get out of sync
\end{itemize}

\section{Future Improvements}
\begin{itemize}
    \item Better handling of complex shell commands
    \item Automatic recovery from errors
    \item More efficient token usage
    \item Improved command batching
\end{itemize}

\end{document}